\section{Preliminaries}
\label{sec:prelim}

		%%%%%%%%%%%%%%%%%%%% IMAGE %%%%%%%%%%%%%%%%%%%%
		\DefFig{fig:mis}{preliminaries/mississippi}{0.55}
			{Array representation of	`mississippi' string.}
			
Throughout this paper we represent the string we are searching in as $S$ of 
length 
$|S| = N$, and the pattern we are searching for as $P$. $S[i]$ represents the 
symbol located at position $i$ in $S$, and $S[i..j]$ represents the substring of 
$S$ beginning at position $i$ and ending at $j$ inclusive. Strings are 
one-based, so in Figure \ref{fig:mis} $S[1] = $ `m', $S[3] = $ `s', and $S[1..3] 
= $ `mis'.

The $i^{th}$ \emph{suffix} is thus defined as $S[i..N]$, so the $1^{st}$ suffix 
in Figure \ref{fig:mis} is $S[1..12] = $ `mississipi\$', and the $5^{th}$ suffix 
is $S[5..12] = $ `issippi\$'. The $i^{th}$ prefix is defined as $S[1..i]$, so
the $5^{th}$ prefix in Figure \ref{fig:mis} is $S[1..5] = $ `missi'.

The $log$ operation is base 2 unless otherwise stated.

\subsection{Background}
In 1970 Knuth, Morris and Pratt (KMP) discovered an algorithm to match patterns 
in time proportional to the length of the text \cite{knuth1977, 
mccreight1976}. If the text is large, then KMP becomes ineffective for ranking 
and pattern discovery. Moreover, KMP is only useful for exact matches.

One alternative to KMP for document ranking is the use of an inverted index, but
they must work with keywords and are thus inappropriate for many applications,
such as searches on certain oriental languages, and other strings that do not
have a clear definition of keywords (MIDI, for example). Suffix arrays are also 
more efficient than inverted files for searching phrases or partial 
patterns \cite{marin2003, hon2009}. This was originally possible with the suffix 
tree~\cite{weiner1973, mccreight1976} (a precursor to the suffix array), 
although suffix trees require three to five times as much 
space~\cite{manber1993}.


\subsection{Suffix Arrays}
In its simplest form, a suffix array can be constructed for a string
$S[1..N]$ like so:

\begin{enumerate}
	\item
		Construct an array of pointers to all suffixes $S[1..N]$, 
		$S[2..N]$, ..., $S[N..N]$.
	\item
		Sort these pointers by the lexicographical (i.e. alphabetical) ordering 
		of their associated suffixes.
\end{enumerate}

Figure \ref{fig:mis} shows an example string, `mississippi'. The construction of 
the corresponding suffix array is shown in Figure \ref{fig:sa-make-mis}.
Construction of suffix arrays is now a well studied problem \cite{puglisi2007}.

\newpage
		%%%%%%%%%%%%%%%%%%%% IMAGE %%%%%%%%%%%%%%%%%%%%
		\DefFig{fig:sa-make-mis}{preliminaries/mississippi-sa-sort}{0.65}
			{Construction of Suffix Array for `mississippi'.}

		%%%%%%%%%%%%%%%%%%%% IMAGE %%%%%%%%%%%%%%%%%%%%
		\DefFig{fig:sa-bwt-mis}{preliminaries/mississippi-sa}{0.35}
			{Suffix Array and Burrows-Wheeler Transform for
			`mississippi' string.}

\subsection{Burrows-Wheeler Transform}
The Burrows-Wheeler Transform (BWT) is a string of length $N$ defined by the suffix array $SA$ and the original text $S$. In particular, $BWT[i] = S[SA[i]-1]$, and $BWT[1] = $ `\$', that is, the $i^{th}$ symbol of the BWT is the symbol prior to the $i^{th}$ suffix in the Suffix Array $SA$. See Figure \ref{fig:sa-bwt-mis}.

As proposed by Ferragina and Manzini, when a BWT is stored alongside a Suffix Array, it is known as a \emph{FM-Index} \cite{ferragina2000}, which supports backwards search.

\subsection{Rank Query}
Munro \cite{munro1996} describes how to do rank queries on binary strings in $O(1)$ time using $o(n)$ bits of extra space. Much of the early work on rank
queries focussed on binary strings.

A rank query on the string $S$ is 
defined as $rank_S(i, c) = n $, with $n$ being the number of times symbol $c$ 
appears in the range $S[1, i]$. This paper omits 
the subscript when the string we are querying is clear from the context. For 
example in Figure \ref{fig:rank-mis}, $rank(9, s) = 3$. If $i \le 0$ then 
$rank(i, c) = 0$.

		%%%%%%%%%%%%%%%%%%%% IMAGE %%%%%%%%%%%%%%%%%%%%
		\DefFig{fig:rank-mis}{preliminaries/rankquery}{0.6}
			{Rank query $rank(9, s) = 3$ on the Burrows-Wheeler Transform
			of `mississippi'.}
\newpage
\subsection{Backward Search}
Since any pattern $P$ in $S$ is a prefix of a suffix, and because the suffixes 
are lexicographically ordered, all occurrences of a search pattern $P$ lie in a 
contiguous portion of the 
Suffix Array. In earlier implementations, the range that this 
pattern lies on would be located by using successive binary searches. Backward 
search utilises the BWT 
in a series of paired rank queries, improving the query performance 
considerably ~\cite{claude2008, ferragina2009, ferragina2007, golynski2006, 
makinen2007b, makinen2007a, marin2003, navarro2006}

Backward search issues $|P|$ pairs of rank queries, where $|P|$ 
denotes the length of the pattern. The paired rank queries are:

					$$ s' = C[P[i]] + rank(s - 1, P[i]) + 1$$
					$$ e' = C[P[i]] + rank(e, P[i])$$

Where $s$ denotes the start of the range, initially at $s = 1$, and $e$ is the 
end of the range, initially $e = N$.

In Figure \ref{fig:bws-1} there is a column $F$, which contains the first 
symbol for each suffix. Note that the $F$ column is not stored as we store it 
encoded as $C$ instead.

$C$ is an array\footnote{Note that we are indexing $C$ by a symbol $P[i]$, so 
this may be implemented with a suitable hash function.} containing the count of 
all symbols in $\Sigma$ which sort lexicographically before $P[i]$, where 
$\Sigma$ is the alphabet from our original string $S$, as in Figure 
\ref{fig:c-tab}.

In the first iteration 
we query the final character of the pattern, so $i = |P|$.  For each 
iteration, we decrement $i$ until $i = 1$. This maintains the invariant that
$SA[s..e]$ contains all the suffixes of which $P[i..|P|]$ is a prefix, and hence
all locations of $P[i..|P|]$ in $S$. This is illustrated in Figure 
\ref{fig:bws-2} through to Figure \ref{fig:bws-4}. If at any stage $e < 
s$, then the pattern does not exist in our original string.

An example is given in Figure \ref{fig:bws-1} through to Figure \ref{fig:bws-4},
where the pattern \emph{`iss'} is searched for in the string `mississippi',
starting with $i = 3$, $P[3] = `s'$. The working for each rank query is shown 
below each figure. We represent the current symbol as $c$ to avoid confusion 
between `s' and $s$ and $s'$.


		%%%%%%%%%%%%%%%%%%%% IMAGE %%%%%%%%%%%%%%%%%%%%
		\DefFig{fig:c-tab}{preliminaries/C-table}{0.3}
			{Table $C$ of number of occurrences in $F$ of each symbol which
			sorts alphabetically before the displayed symbol.}

\clearpage			
		%%%%%%%%%%%%%%%%%%%% IMAGE %%%%%%%%%%%%%%%%%%%%
		\DefFig{fig:bws-1}{preliminaries/bwt-1}{0.5}
			{First stage of backwards search for `iss' on `mississippi'
			 string - before any rank queries have been made.}

\begin{enumerate}
	\item
		Starting from $s = 1$ and $e = 12$ as in Figure \ref{fig:bws-1},
		and $c = P[i] = $`s' where $i = 3$, we make our first two rank queries:
			$$s' = C[c] + rank(0, c) + 1 = 8 + 0 + 1 = 9$$
			$$e' = C[c] + rank(12, c) = 8 + 4 = 12$$
		
			%%%%%%%%%%%%%%%%%%%% IMAGE %%%%%%%%%%%%%%%%%%%%
			\DefFig{fig:bws-2}{preliminaries/bwt-2}{0.5}
				{Second stage of backwards search for `iss' on `mississippi'
				string. All the occurrences of `s' lie in $SA[9..12]$.}
\clearpage				
	\item
		From $s = 9$ and $e = 11$ as in Figure \ref{fig:bws-2},
		and $c = P[i] = $`s' where $i = 2$, our next two rank queries are:
			$$s'' = C[c] + rank(8, c) + 1 = 8 + 2 + 1 = 11$$
			$$e'' = C[c] + rank(12, c) = 8 + 4 = 12$$
	
			
			%%%%%%%%%%%%%%%%%%%% IMAGE %%%%%%%%%%%%%%%%%%%%
			\DefFig{fig:bws-3}{preliminaries/bwt-3}{0.5}
				{Third stage of backwards search for `iss' on `mississippi'
				string. All the occurrences of `ss' lie in $SA[11..12]$.}
	
	\item
		From $s = 11$ and $e = 12$ as in Figure \ref{fig:bws-3},
		and $c = P[i] = $`i' where $i = 1$, our final two rank queries are:
			$$s''' = C[c] + rank(10, c) + 1 = 1 + 2 + 1 = 4$$
			$$e''' = C[c] + rank(12, c) = 1 + 4 = 5$$
			
			
			%%%%%%%%%%%%%%%%%%%% IMAGE %%%%%%%%%%%%%%%%%%%%
			\DefFig{fig:bws-4}{preliminaries/bwt-4}{0.5}
				{Fourth and final stage of backwards search for `iss' on
				`mississippi' string. All the occurrences of `iss' lie in
				$SA[4..5]$.}
				
\end{enumerate}
