\section{Preliminaries}
This section will discuss previous data structure development.

\subsection{Suffix Arrays}
In its simplest form, a suffix array can be constructed for a string
$S[0..N-1]$ - where the first character of $S$ is $S[0]$, and $N$ is the
length of the string - like so:

\begin{enumerate}
	\item
		Construct an array of pointers to all suffixes ($S[0..N-1]$, 
		$S[1..N-1]$, $S[2..N-1]$, and so on...) in the string,
	\item
		Sort these pointers by the lexicographical ordering of their associated
		suffixes
\end{enumerate}

Take for example the string `mississippi' in Figure \ref{fig:mis}.
The construction of the corresponding suffix array is shown in Figure
\ref{fig:sa-make-mis}.


 		%%%%%%%%%%%%%%%%%%%% IMAGE %%%%%%%%%%%%%%%%%%%%
		\DefFig{fig:mis}{preliminaries/mississippi}{0.55}
			{Array representation of	`mississippi' string}

		%%%%%%%%%%%%%%%%%%%% IMAGE %%%%%%%%%%%%%%%%%%%%
		\DefFig{fig:sa-make-mis}{preliminaries/mississippi-sa-sort}{0.65}
			{Construction of Suffix Array for `mississippi'}



\subsection{Burrows-Wheeler Transform}
The Burrows-Wheeler Transform (BWT) is calculated by $BWT[i] = S[SA[i]-1]$, and $BWT[0] = \$$ (that is, it is the character previous to the suffix), as in Figure \ref{fig:sa-bwt-mis}.

When the BWT is stored in a Wavelet Tree for rank queries, this becomes what is known as a `FM-index', which enables backwards search.


		%%%%%%%%%%%%%%%%%%%% IMAGE %%%%%%%%%%%%%%%%%%%%
		\DefFig{fig:sa-bwt-mis}{preliminaries/mississippi-sa}{0.35}
			{Suffix Array and Burrows-Wheeler Transform for
			`mississippi' string}


\subsubsection{Rank Query} % TODO: REFERENCE THIS SECTION?
A Rank Query is defined as $rank(i, c) = n $, with $n$ being the number of
times symbol $c$ appears on the range $[0, i]$. For example in Figure
\ref{fig:rank-mis}, $rank(9, s) = 3$.

		%%%%%%%%%%%%%%%%%%%% IMAGE %%%%%%%%%%%%%%%%%%%%
		\DefFig{fig:rank-mis}{preliminaries/rankquery}{0.6}
			{Rank query $rank(s, 8)$ on `mississippi' BWT}

\subsubsection{Backwards Search} % TODO: AND THIS ONE
Backwards search involves a series of rank queries on the BWT to locate the
range in the Suffix Array that contains all the occurrences of the search
pattern $P$. For $|P| = m$, $m$ pairs of rank queries will be issued.

For example, if we are searching for the pattern $P = $ \emph{`iss'} on the
string `mississippi', we initiate the process from the final character of $P$: 
$P[2] = s$. The rank queries we issue are:

					$$ s' = C[P[i]] + rank(P[i], s)$$
					$$ e' = s' + (rank(P[i], e) - rank(P[i], s))$$

Where $s$ is the start of the range, $e$ is the end of the range and $C$ is a
table of initial positions for each symbol, as in Figure \ref{fig:c-tab}. An
example is given in Figure \ref{fig:bws-1} through to Figure \ref{fig:bws-4}.
Note that the `F' column isn't stored as we store $C$ instead.


		%%%%%%%%%%%%%%%%%%%% IMAGE %%%%%%%%%%%%%%%%%%%%
		\DefFig{fig:c-tab}{preliminaries/C-table}{0.2}
			{Table of counts and first indexes of each symbol}

		% TODO: COMBINE 2 steps into one image: 1st + 2nd, 3rd + 4th
		%%%%%%%%%%%%%%%%%%%% IMAGE %%%%%%%%%%%%%%%%%%%%
		\DefFig{fig:bws-1}{preliminaries/bwt-1}{0.4}
			{First stage of backwards search for `iss' on `mississippi'
			 string - before any rank queries have been made}

		%%%%%%%%%%%%%%%%%%%% IMAGE %%%%%%%%%%%%%%%%%%%%
		\DefFig{fig:bws-2}{preliminaries/bwt-2}{0.4}
			{Second stage of backwards search for `iss' on `mississippi'
			string. All the occurrences of `s' lie in $SA[s..e]$}

		%%%%%%%%%%%%%%%%%%%% IMAGE %%%%%%%%%%%%%%%%%%%%
		\DefFig{fig:bws-3}{preliminaries/bwt-3}{0.4}
			{Third stage of backwards search for `iss' on `mississippi'
			string. All the occurrences of `ss' lie in $SA[s..e]$}

		%%%%%%%%%%%%%%%%%%%% IMAGE %%%%%%%%%%%%%%%%%%%%
		\DefFig{fig:bws-4}{preliminaries/bwt-4}{0.4}
			{Fourth and final stage of backwards search for `iss' on
			`mississippi' string. All the occurrences of `iss' lie in
			$SA[s..e]$}
		\newpage


\subsection{Binary Wavelet Trees}
Binary Wavelet Trees encode the BWT as a tree of bit-vectors, to enable
$O(\log \sigma)$ - where $\sigma$ is the size of the alphabet - rank queries. This procedure is followed:

\begin{enumerate}
    \item
		Encoding half the alphabet as 0, and half as 1 as in Figure
    	%\ref{fig:wt-enc-bwt}, for example:
    	$\Sigma = \{ \$, i, m, p, s \} : enc(\Sigma) = \{ 0, 0, 0, 1, 1 \}$
    \item
		Group each 0-encoded symbol, $\{ \$, i, m \}$, as a sub-tree
    \item
		Group each 1-encoded symbol, $\{ p, s \}$, as a sub-tree
    \item
		Reapply to each sub-tree recursively until there is only one symbol
    	left
\end{enumerate}

For a more detailed example see Figure %\ref{fig:bin-wt-pp}.


		%%%%%%%%%%%%%%%%%%%% IMAGE %%%%%%%%%%%%%%%%%%%%
		\DefFig{fig:wt-enc-bwt}{preliminaries/encodebwt}{0.5}
			{Root node of Binary Wavelet Tree encoding for `mississippi' BWT}


After this, a rank query can be done by $\log N$ popcounts\footnote{A
$popcount(i)$ operation counts all $1$s up to a point $i$. To count $0$s, this
number is subtracted from $i$} on the bit-vectors. For example, if we wanted to 
know $rank(5, e)$ in Figure binary-wt, we know that $enc(e) = 0$ at this level, so:

\begin{enumerate}
    \item
		We count the number of $0$s on $bitvector[0..5]$, which is $4$, this
   		gives us the index to query in our 0-child (although it is $3$, not $4$ 
		due to zero-basing);
    \item
		We query $popcount(3) = 2$, as $e$ is now encoded as $1$. We traverse
    	the 1-branch this time.
    \item
		$popcount(1) = 2$, which we use as the index in the child on the
    	1-branch.
    \item
		$popcount(1) = 0$, which we subtract from two, as $e$ is encoded as
    	$0$ here, and hence get $2$. Since our children at this point are leaf
    	nodes, we return $2$ as our result.
\end{enumerate}

Hence $rank(5, e) = 2$. Popcounts are fast - $O(1)$ - on these bit-vectors due
to RRR.


\subsection{Multi-ary Wavelet Trees}
Similar to a Binary Wavelet Tree, although now we encode it like so:

\begin{enumerate}
    \item Encoding one $N^{th}$ of the alphabet as $0$, the next $N^{th}$ as $1$,
    the next $N^{th}$ as $2$ and so on...
    \item Group each 0-encoded symbol as a sub-tree
    \item Group each 1-encoded symbol as a sub-tree
    \item Group each 2-encoded symbol as a sub-tree, and so on until $N-1$
    \item Reapply to each sub-tree recursively until there is only one symbol
    left
\end{enumerate}
For an example see Figure .... It is clear that popcounts will not work
here like they do for binary, and hence the RRR requires modification too.

		%%%%%%%%%%%%%%%%%%%% IMAGE %%%%%%%%%%%%%%%%%%%%
		\DefFig{fig:bin-wt-pp}{preliminaries/binaryWT}{1}
			{Binary Wavelet Tree for `Peter Piper...' with alphabet
			encodings at each level for the left-most branch}
		
		\DefFig{fig:multi-wt-pp}{preliminaries/multiaryWT}{1}
			{Multi-ary Wavelet Tree for `Peter Piper...' with alphabet
			encodings at each level for the left-most branch.}



\subsection{RRR}
To construct the RRR we divide the bit-vector into several so called \emph{superblocks}, we then divide these superblocks into further \emph{blocks}, as in Figure \ref{fig:blocks}.

For each of these blocks we store a class number $c$, which is the number of $1$s in the block. This is used as a lookup key in a table $G$, which is a table of tables, and will be explained shortly. We also store offset $o$, which is an index into the table at $G[c]$. See Figure% \ref{fig:classes}.

$G$ is a table where $G[c]$ is a sub-table containing an array of cumulative sums for each respective bit, on all permutations of $c$ 1-bits. It is important to note that the size of $o$ varies, due to it being an offset into $G[c]$, where $|G[c]| = b \choose c$, and hence can be encoded in $\log b \choose c$ bits. See Figure \ref{fig:bin-gtab} .

RRR allows us to calculate popcounts in $O(1)$ time, and offers compression of our Wavelet Tree. Here we are using a binary RRR structure, but it is possible to support multi-ary WTs with a more extensive class allocation, for example on an alphabet $\Sigma = \{ A, B, C, D \}$ classes could be a 3-tuple of counts for $|\Sigma| - 1$ symbols. The G-table entries would have to hold more than one cumulative sums, too. We would also need to adjust the hashing algorithm for the lookup, but an upper bound is known, and hence a perfect hashing function is possible.

		\DefFig{fig:blocks}{preliminaries/blocks}{0.8}
			{Block division scheme for `Peter Piper...' Wavelet Tree's root
			node bit-vector}

		\DefFig{fig:rrr-seq}{preliminaries/classoffsets}{0.8}
		{RRR Sequence with three superblocks and the blocks for which they
		maintain intermediate counts}

		\DefFig{fig:bin-gtab}{preliminaries/binary-g-table}{0.75}
			{Binary RRR Count Table, with example lookup for class $c = 2$
			and offset $o = 3$ in a RRR sequence}

		\DefFig{fig:gen-gtab}{preliminaries/multi-g-table}{1}
			{`Four-ary' RRR Count Table, with example lookup for class $c = 2$, 
			which represents $(2, 2, 0, 1)$, and offset $o = 3$ in a RRR
			sequence}
			
\subsection{Multiary Wavelet Trees}
\begin{itemize}
\item
  Uses General RRR to create a shallower WT
\end{itemize}
We also implement and test some other variations.



\subsubsection{Multiary Wavelet Tree Variation 1 : Uncompressed}
\begin{itemize}
\item
  Concatenated bitmaps for each symbol
\item
  rank becomes a count on a range
\end{itemize}



\subsubsection{Multiary Wavelet Tree Variation 2 : Multi-binary RRR}
\begin{itemize}
\item
  As above, bitmaps are created for each symbol
\item
  these bitmaps are stored in a binary RRR, as implemented by
  Francisco Claude.
\end{itemize}