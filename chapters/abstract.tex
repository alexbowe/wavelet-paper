Self-indexes for pattern matching based on Suffix
Arrays and the \\ Burrows-Wheeler Transform have recently grown increasingly 
popular.
The fundamental operation in these self-indexes is the rank-query; $rank(i, c)$ 
requests the 
number of occurrences of symbol $c$ before position $i$ in strings. The Wavelet 
Tree is the data structure of choice for implementing these rank queries. 
Currently, Wavelet Tree implementations encode the Burrows-Wheeler Transform in 
binary, which they then store as RRR sequences;
a structure offering $O(1)$ rank queries and zeroth-order entropy compression on 
binary strings. A generalisation of this extends Wavelet Trees to have higher 
order encoding, that is increased branching, in theory making traversal faster. 
To support the implementation of a Multiary
Wavelet Tree, this thesis investigates the generalisation of RRR to sequences 
over small alphabets. We also analyse the use of concatenated bitmaps to 
represent sequences over small alphabets, thus allowing continued use of the
binary RRR structure. Our results show that Multiary Wavelet Trees are faster
than their binary counterparts, but have memory issues in the case of the
Generalised RRR. We also show a practical alternative to be binary RRR on concatenated bitmaps.