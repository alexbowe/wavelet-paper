Self-index data structures for pattern matching based on Suffix
Arrays and the \\ Burrows-Wheeler Transform have recently grown
popular.
The fundamental operation in these self-indexes is the rank-query; $rank(i, c)$ 
requests the 
number of occurrences of symbol $c$ before position $i$ in a string. The Wavelet 
Tree is the data structure of choice for implementing these rank queries. 
Currently, Wavelet Tree implementations encode the Burrows-Wheeler Transform as 
a hierarchy of binary strings, which they then store as RRR sequences;
the RRR structure offers $O(1)$ rank queries and zeroth-order entropy compression for 
binary strings. A generalisation of this extends Wavelet Trees to have higher 
order encoding, that is increased branching, in theory making traversal faster. 
To support the implementation of such a Multiary
Wavelet Tree, this thesis investigates the generalisation of RRR to sequences 
over small alphabets. We also analyse the use of concatenated bitmaps to 
represent sequences over small alphabets, thus allowing continued use of the
binary RRR structure. Our results show that Multiary Wavelet Trees are faster
than their binary counterparts, but have require large amounts of memmory in the 
case of the Generalised RRR. We also show binary RRR on concatenated bitmaps to 
be a practical alternative.