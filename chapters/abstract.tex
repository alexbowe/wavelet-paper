In recent years, self-indexes for pattern matching based on Suffix
Arrays and the Burrows-Wheeler Transform (BWT) have grown
increasingly popular. The fundamental operation on these is the
rank-query; $rank(i, c)$ requests the number of occurrences of
symbol $c$ before position $i$ in strings. This is best provided by
a Wavelet Tree (WT). Currently, WT implementations encode the BWT
in binary. A generalisation of this extends WTs to higher order
encoding, i.e. multi-arity, which reduces the depth of a WT, in
theory making traversal faster. This paper aims to investigate the
benefits of multi-ary WTs in practice, and whether specific
properties of the BWT induce space-savings in the WT.
