\section{Experiments}
We will detail below what our expected findings were, and our method, including 
the data selected for the experiments and why it was chosen.

\subsection{Hypotheses}
The experiments were designed to test these hypotheses:

\begin{itemize}
    \item 	
			Since BWTs have many runs of the same symbol, RRR classes and 
			offsets will not be equally distributed

    \item 	
			Increasing the arity of a Wavelet Tree will make it shallower,
		   	hence reducing the amount of nodes a query must visit, resulting
		   	in faster queries
			
    \item  
			There is a practical limit to the order of the arity increase for 
		   	Generalized RRR, since the count table will increase in size.
\end{itemize}

\subsection{Method}
For each Multiary Wavelet Tree variant (using Generalised RRR, Multi-binary
RRR, and simple concatenated bitmaps to provide
sequence ranking at each node which we call ``simple''), we generated 1000 
random rank queries $rank(c, i)$. For three runs the mean query time was 
recorded, and the minimum result was taken as our result, to reduce interference 
from external factors. We repeated this as we doubled the arity.

The size of  the Wavelet Tree (including the encoded sequences at each node) was 
recorded. The size of the RRR count tables was recorded for the cases which used 
a variant of RRR. For the Generalized RRR, we recorded how many unique class and 
offset values were encountered, and calculated the percentage of total possible 
classes and offsets these were.

We used Francisco Claude's SPIRE 20... (FIXME)\footnote{Claude's compressed data 
structures library is available at http://libcds.recoded.cl} implementation as a 
base line for comparison. We used the same default block-size and 
super-block factor as Claude.

The data set is described below\footnote{The BWT files for each dataset are 
available at\\ http://bwt-corpus.s3.amazonaws.com/list.html, and are 
reconstructible using the scripts at http://github.com/alexbowe/bwt-corpus}.

The experiments were run on Mac OS X Snow Leopard with a 2.4 GHz Intel Core 2
Duo processor, and 4GB 1067 MHz DDR3 RAM.

\subsubsection{Test Data}
Since Wavelet Trees are used to provide faster rank queries on BWTs, we
constructed BWT strings over a selection of texts taken from the \emph{Pizza\&Chili}
Corpus website\footnote{http://pizzachili.dcc.uchile.cl}. This is the standard
corpus used when developing compressed self indexes, and has been collected by
Paolo Ferragina and Gonzalo Navarro, two prominent contributors of suffix array 
research.

The corpus consists of source code for the Linux kernel and GNU C Compiler
(GCC), protein sequences, DNA, English texts from Project
Gutenberg\footnote{http://www.gutenberg.org}, and XML-formatted bibliographies
of from several major Computer Science journals. These are considered to be
representative of the sort of problems a suffix array may be used for. In the 
case of the English corpus, we also took a word-based integer mapping, allowing 
us to test word-based indexing.

Three data sizes were used to test the scalability: 25MB, 50MB and 75MB. 
These were selected as to not be in cache entirely, but not take large amounts 
of time for experimentation. The length and alphabet size of these files are 
described in Table \ref{tab:files}.

\begin{center}
\begin{table}[h]
\begin{tabular}{crrrrrr}
\toprule
\multirow{2}{*}{Data} & \multicolumn{2}{c}{25 MB} & \multicolumn{2}{c}{50 MB} &  
	\multicolumn{2}{c}{75 MB}\\
		  \cmidrule(r){2-7}
	      &\multicolumn{1}{c}{$\sigma$}& \multicolumn{1}{c}{length}
		  &\multicolumn{1}{c}{$\sigma$}&\multicolumn{1}{c}{length}
		  &\multicolumn{1}{c}{$\sigma$}&\multicolumn{1}{c}{length}\\
\midrule
xml 	  & 96	   & 26214400 & 98 	   & 52428804 & 98 	   & 78643208 \\
dna 	  & 14     & 26214400 & 19 	   & 52428804 & 21     & 78643208 \\
proteins  & 25     & 26214400 & 29     & 52428804 & 33     & 78643208 \\
sources   & 116    & 26214400 & 229    & 52428804 & 229    & 78643208 \\
english   & 154    & 26214400 & 179    & 52428804 & 188    & 78643208 \\
words     & 83083  & 5969593  & 115754 & 11860943 & 164757 & 17668587 \\
\bottomrule
\end{tabular}
\caption{Text lengths and alphabet sizes for each test file}
\label{tab:files}
\end{table}
\end{center}

\newpage
\subsection{Results}
A table of results\footnote{The preprocessed data is available at
http://github.com/alexbowe/multiary-wt} is available in Appendix A. \emph{Please 
note that 
measurements for the Generalized RRR with arity 16 are missing, since memory 
constraints meant the experiments took too long due to paging. Also, note that
the simple wavelet tree timings have been plotted on a different graph due to
the difference in scale.}

Table of classes/offsets encountered.!!

		\DefFig{fig:stime-eng-75}{experiments/simple_time_english_75MB}{1}
			{Query Times for Simple Multiary Wavelet Tree of increasing arity
			for the 75MB \emph{english} file}
			
		\DefFig{fig:time-eng-75}{experiments/time_english_75MB}{1}
			{Query Times for Non-simple Wavelet Trees of increasing arity
			for the 75MB \emph{english} file}
			
		\DefFig{fig:mem-eng-ints-75}{experiments/mem_english_ints_75MB}{1}
			{Memory consumption for Wavelet Trees of increasing arity for
			the 75MB \emph{words} file}
			
		\DefFig{fig:mem-eng-75}{experiments/mem_english_75MB}{1}
			{Memory consumption for Wavelet Trees of increasing arity for
			the 75MB \emph{english} file}
			
		\DefFig{fig:mem-prot-75}{experiments/mem_proteins_75MB}{1}
			{Memory consumption for Wavelet Trees of increasing arity for
			the 75MB \emph{proteins} file}

\subsection{Observations}
It can be seen that ... time doesnt increase for 
non-simple wavelet trees when the corpus is larger.

Although the wavelet tree remains compressed for the generalized RRR, the RRR 
table grows considerably. Although our implementation sacrifices some of the
compression techniques of Claude's for sparsity, note that in the binary case 
the RRR table is negligible.

The size of the multi-RRR at arity 2 is higher, because the optimisation isn't 
made whereby you don't have to store the rank structure for the 0-bitmap.