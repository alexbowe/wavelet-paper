\section{Experiments}
\label{sec:experiments}

We will detail below what our expected findings were, and our method, including 
the data selected for the experiments and why it was chosen.

\subsection{Hypotheses}
The experiments were designed to test the following hypotheses:

\begin{itemize}
    \item 	
			Since BWTs have many runs of the same symbol, RRR classes and 
			offsets will not be equally distributed.

    \item 	
			Increasing the arity of a Wavelet Tree will make it shallower and
		   	hence reduce the amount of nodes visited per query, resulting
		   	in faster queries.
			
    \item  
			There is a practical limit to the order of the arity increase for 
		   	Generalized RRR, since the RRR count table will increase in size. In
			particular there is tension between the size of the RRR 
			count table and the size of the class / offset sequences.
\end{itemize}

\subsection{Method}
For each Multiary Wavelet Tree variant\footnote{All source code is available at 
http://github.com/alexbowe/multiary-wt.} (using Generalised RRR, Multi-binary
RRR, and uncompressed concatenated bitmaps to provide
sequence ranking at each node), we generated 1000 
random rank queries $rank(c, i)$. For three runs the mean query time was 
recorded, and the minimum result was taken as our result, as it is the time 
least influenced by external factors (e.g. Operating System swapping). The above
experiment was repeated as we doubled the arity.

The size of  the Wavelet Tree (including the RRR encoded sequences at each node) 
was recorded. The size of the RRR count tables was recorded for the cases which 
used a variant of RRR. For the Generalized RRR, we recorded how many unique 
class and offset values were encountered, and calculated the percentage of total 
possible 
classes and offsets these were\footnote{All raw and processed data is 
available at http://github.com/alexbowe/multiary-wt, and the graphs are 
available at http://github.com/alexbowe/wavelet-paper/tree/thesis/.}.

We used Francisco Claude's SPIRE 2008 implementation of binary 
RRR\footnote{Claude's compressed data 
structures library is available at http://libcds.recoded.cl} implementation as a 
base line for comparison \cite{claude2008}. We used the same default block-size 
and super-block factor as Claude, which were 15 and 32 respectively.

The data set used for testing is described below\footnote{The BWT files for each 
dataset are 
available at\\ http://bwt-corpus.s3.amazonaws.com/list.html, and are 
reconstructible using the scripts at http://github.com/alexbowe/bwt-corpus.} in 
Section \ref{sec:data} with some statistics in Table \ref{tab:files}.

The experiments were run on an otherwise idle Mac OS X Snow Leopard with a 2.4 
GHz Intel Core 2 Duo processor, and 4GB 1067 MHz DDR3 RAM.

\pagebreak
\subsection{Test Data}
\label{sec:data}
Since a prime use of Wavelet Trees is to provide faster rank queries over BWTs 
(as part of an FM-Index), we
constructed BWT strings over a selection of texts taken from the 
\emph{Pizza\&Chili}
Corpus website\footnote{http://pizzachili.dcc.uchile.cl}. This is the standard
corpus used when developing compressed self indexes, and has been collected by
Paolo Ferragina and Gonzalo Navarro, two prominent contributors of suffix array 
research.

The corpus consists of source code for the Linux kernel and GNU C Compiler
(GCC), protein sequences, DNA, English texts from Project
Gutenberg\footnote{http://www.gutenberg.org}, and XML-formatted bibliographies
from several major Computer Science journals. These are considered to be
representative of the sort of texts a suffix array may be used for. In the 
case of the English corpus, we also took a mapping of each unique word to an 
integer, allowing us to test word-based indexing.

Three data sizes were used to test the scalability: 25MB, 50MB and 75MB. 
Importantly, these data sizes are much bigger than the available CPU cache, but
will not take large amounts of time for experimentation. The length and alphabet 
size of these files are described in Table \ref{tab:files}.

\begin{center}
\begin{table}[h]
\begin{tabular}{crrrrrr}
\toprule
\multirow{2}{*}{Data} & \multicolumn{2}{c}{25 MB} & \multicolumn{2}{c}{50 MB} &  
	\multicolumn{2}{c}{75 MB}\\
		  \cmidrule(r){2-7}
	      &\multicolumn{1}{c}{$\sigma$}& \multicolumn{1}{c}{length}
		  &\multicolumn{1}{c}{$\sigma$}&\multicolumn{1}{c}{length}
		  &\multicolumn{1}{c}{$\sigma$}&\multicolumn{1}{c}{length}\\
\midrule
xml 	  & 96	   & 26 214 400 & 98 	   & 52 428 804 & 98 	   & 78 643 208 \\
dna 	  & 14     & 26 214 400 & 19 	   & 52 428 804 & 21     & 78 643 208 \\
proteins  & 25     & 26 214 400 & 29     & 52 428 804 & 33     & 78 643 208 \\
sources   & 116    & 26 214 400 & 229    & 52 428 804 & 229    & 78 643 208 \\
english   & 154    & 26 214 400 & 179    & 52 428 804 & 188    & 78 643 208 \\
words     & 83 083  & 5 969 593  & 115 754 & 11 860 943 & 164 757 & 17 668 587 \\
\bottomrule
\end{tabular}
\caption{Text lengths and alphabet sizes for each test file.}
\label{tab:files}
\end{table}
\end{center}
