\section{Conclusion}
\label{sec:conclusion}

From our observations, we have discovered that in the case of documents that fit 
in memory, the RRR count table expands too rapidly to make increasing the arity 
worthwhile, unless its growth can be addressed (see Section \label{sec:future}).

Something about sparsity...

Since the RRR count table is shared among all nodes and even all Wavelet Trees
of the same or smaller arity and blocksize, it may be the case that when 
documents are significantly large, the overhead of the RRR count table becomes 
negligible. For such documents, a distributed approach may be required. The more 
documents we index, the less significant the table size will be.

However, in the case of smaller documents, we have shown that there are simple 
ways to implement Multiary Wavelet Trees using rank structures for binary 
alphabets. In the case of our `Multi-Binary RRR' Wavelet Tree, we saw rank 
queries become faster, while the Wavelet Tree nodes didn't grow too large. The 
table didn't grow, and as such...

\section{Future Work}
\label{sec:future}
There are several promising avenues for future work which this thesis has helped
reveal:

\begin{enumerate}
\item
	Investigate if there is any way to make the count table for Generalized RRR 
	smaller. One such idea might be to only store base counts which can
	generate all cyclic permutations of a block.
	
\item
	Similar to above, another option is to share count table entries among 
	different blocks which have similar positioning but for different symbols. 
	For example the block $b_1 = [0, 1, 2, 2, 3]$ has the same count table entry 
	for $c = 2$ as the block $b_2 = [0, 2, 1, 1, 0]$ does for $c = 1$.
	
\item
	Implement and investigate a Multiary Huffman-Shaped Wavelet Tree (see 
	M\"{a}kinen's work for details on Huffman-Shaped Wavelet
	Trees~\cite{huffmanWT:makinen2005}). This may
	overcome the count table issue while still reducing the tree depth.

\item
	There may be certain blocks which are queried more often, so we may only
	need to keep a certain percentage of those blocks in memory. The other 
	blocks may be stored on disk and loaded into memory on the occasion that 
	they are queried.

\item
	Distribution of the RRR table among nodes in a cluster, allowing it to be
	held in memory as restricted by the cluster as a whole, not a single
	computer. Then, increasing arity would be an issue of how many nodes are
	on the cluster.

\item
	Investigate Multiary Wavelet Trees which use the bitmap concatenation 
	technique, but encode each node using other binary sequence structures which 
	answer rank queries, such as a Sadakane Array.

\end{enumerate}