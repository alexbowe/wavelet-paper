\section{Conclusion}
\label{sec:conclusion}

From our observations we have discovered that there is sparsity in the classes 
and block permutations encountered in our test data, and it 
can have a significant effect on the supporting table size, in
the case of Generalised RRR. However, making use of this sparsity dynamically
requires pointers, and may be the cause of some time overhead and cache misses 
while querying.

Since the RRR count table is shared among all nodes and even all Wavelet Trees
of the same or smaller arity and blocksize, it may be the case that when 
documents are significantly large, or we are indexing a large collection of 
documents, the overhead of the RRR count table becomes negligible. For such 
documents, a distributed approach may be required; we may store the RRR table on
multiple computers, or increase the arity so that the table fits in one central
server for querying by the computers that host the Wavelet Trees. It is likely 
such a configuration may need the extra speed while querying, since it is 
typical of a search engine to host indexes over many servers and 
calculate many queries per second.

For single documents, or small collections of documents, the RRR count table 
expands too rapidly to make increasing the arity worthwhile. We discuss a 
possible method to curb this growth in Section \ref{sec:future}.

However, we have shown that there are simple 
ways to implement Multiary Wavelet Trees using rank structures for binary 
alphabets. In our `Multi-Binary RRR' Wavelet Tree rank 
queries become faster, while the Wavelet Tree nodes did not grow too large. The
Multi-Binary RRR Wavelet Tree was sometimes larger than the original text, 
however.

\section{Future Work}
\label{sec:future}
There are several promising avenues for future work which this thesis has helped
reveal:

\begin{enumerate}
\item
	Investigate if there is any way to make the count table for Generalized RRR 
	smaller. One promising idea is to only store base counts which can
	generate all cyclic permutations of a block. For example, if one block is
	$b_1 = [0, 1, 2, 2, 3]$ and another block is $b_2 = [1, 2, 2, 3, 0]$ (that 
	is, a left cyclic shift of one position applied to $b_1$), then there may be
	a way to calculate the counts for $b_2$ from $b_1$, and no longer store the
	counts for $b_2$.
	
\item
	Similar to above, another option is to share count table entries among 
	different blocks which have similar positioning but for different symbols. 
	For example the block $b_1 = [0, 1, 2, 2, 3]$ has the same count table entry 
	for $c = 2$ as the block $b_2 = [0, 2, 1, 1, 0]$ does for $c = 1$.
	
\item
	Reduce the use of pointers when constructing a sparse RRR table. This may be
	done by completing a full scan of the text before constructing the table,
	and tracking how many unique blocks there are, then allocating them 
	contiguously.

\item
 	Search for a good tradeoff between arity and block size for the
	Generalised RRR; the Generalised RRR Table will grow smaller if a smaller 
	block size is used, but the sequences then become bigger as they have more
	class and offset values (which require less bits, but perhaps not 
	significantly so). This may not be preferable since the table can be shared 
	among many sequences. The analysis of Ferragina et al. \cite{ferragina2007}
	suggests blocksize and arity should be related, in particular blocksize
	$b = \lfloor \frac{1}{2} \log_{A} N \rfloor$, that is, $b$ decreases 
	slightly as arity $A$ increases.

\item
	Implement and investigate a Multiary Huffman-Shaped Wavelet Tree (see 
	M\"{a}kinen's work for details on Huffman-Shaped Wavelet
	Trees~\cite{makinen2005}). This may
	overcome the RRR count table memory consumption while still reducing the 
	tree depth.

\item
	Real world query patterns may show that there are certain blocks which are 
	queried more frequently, so we may only need to keep the RRR count table 
	entries
	for these blocks in memory at all times. The other RRR count table sections 
	may be stored on disk and loaded into memory on the occasion that they are 
	queried.

\item
	Distribute the RRR count table among nodes in a cluster, allowing it to be
	held in memory as restricted by the cluster as a whole, not a single
	computer. Then, increasing arity would be an issue of how many nodes are
	on the cluster. There has been recent research on distributed compressed
	suffix arrays by Russo et al. \cite{russo2010}.

\item
	Investigate Multiary Wavelet Trees which use the bitmap concatenation 
	technique, but encode each node using other binary sequence structures which 
	answer rank queries, such as the rank index by Okanohara and Sadakane 
	\cite{okanohara2006}.

\end{enumerate}