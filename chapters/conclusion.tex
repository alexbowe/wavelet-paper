\section{Conclusion}
\label{sec:conclusion}

From this we have shown that...
\begin{enumerate}
\item
 	The table expands too rapidly to make increasing the arity worthwhile,
	unless this issue can be addressed (see Future Work).
\item
	There are simple ways to implement Multiary Wavelet Trees using rank
	structures for binary alphabets.
\item
	Using the above method, rank queries become faster, but the Wavelet Tree is 
	not necessarily smaller than the original text.
\end{enumerate}

\section{Future Work}
\label{sec:future}
Some future avenues that could be considered include:

\begin{enumerate}
\item
	Investigate if there is any way to make the count table for Generalized RRR 
	smaller. One such idea might be to only store base counts which can
	generate all cyclic permutations.
	
\item
	Implement and investigate a multiary huffman shaped wavelet tree (see 
	M\"{a}kinen's work for details on huffman shaped wavelet
	trees~\cite{huffmanWT:makinen2005}). This may
	overcome the count table issue while still reducing the tree depth.

\item
	The possibility of restructuring a RRR table to allow for more
    efficient caching, thus enabling it to be accessed efficiently from the
	hard drive if it can't fit in memory

\item
	Improved serialisation, for faster import of the structures from disk to
	memory

\item
	Distribution of the RRR table among nodes in a cluster, allowing it to be
	held in memory as restricted by the cluster as a whole, not a single
	computer. Then, increasing arity would be an issue of how many nodes are
	on the cluster.

\item
	Non-linear pattern matching; as a suffix array is similar in functionality
	to a regular expression, what if we were to apply this to a more 
	structured format? For example, searching for structural patterns in code
	for use in refactoring.
\end{enumerate}