\section{Multiary Wavelet Trees and Generalised RRR}
\label{sec:multiary}

This section discusses the design of Multiary Wavelet Trees, and three 
approaches to support rank queries on the nodes.

\subsection{Multiary Wavelet Trees}
Multiary Wavelet Trees are analogous to their binary counterparts, although now 
we encode each node recursively like so:

\begin{enumerate}
    \item Encoding one $A^{th}$ of the alphabet as $0$, the next $A^{th}$ as 
		$1$, the next $N^{th}$ as $2$ and so on until $A-1$. For example, with 
		the `Peter Piper...' string:
		$$\Sigma = \{ \$, \_, P, a, c, d, e, f, i, k, l, o, p, r, s, t \}$$
	   	$$enc(\Sigma) = \{  0,  0, 0, 0, 1, 1, 1, 1, 2, 2, 2, 2, 3, 3, 3, 3 \}$$
    \item Group each 0-encoded symbol as a sub-tree
    \item Group each 1-encoded symbol as a sub-tree
    \item Group each 2-encoded symbol as a sub-tree, and so on until $A-1$
    \item Reapply to each sub-tree recursively until the amount of symbols is
	less than or equal to $A - 1$.
\end{enumerate}

See Figure \ref{fig:multi-wt-pp} for a Multiary Wavelet Tree constructed over
the `Peter Piper...' string.

We can no longer use binary rank queries, and hence Binary RRR, the same way as 
we do with a Binary Wavelet Tree. We discuss three alternative approaches in the 
following sections.

\subsection{Multiary Wavelet Tree Variation 1 : Uncompressed}
		\DefFig{fig:bitmaps}{preliminaries/bitmaps}{0.4}
		{Concatenated bitmap binary encoding of multiary Wavelet Tree Node
		representing `eeecedecfcedee' from the `Peter Piper...' string.}
		
It is possible to represent each encoded symbol $c$, where $c$ is an element of
${ 0, 1,..., A - 1}$ and $A$ is the arity, using $A$ bitmaps. First we construct
the bitmaps for each symbol, as in Figure \ref{fig:bitmaps}, then we concatenate 
these bitmaps and store them as one 
bit-vector. A rank query then involves a ranged binary rank query on $N[c L, c L 
+ i]$ at each node $N$, where $L$ is the length of the node string before 
concatenation, and $i$ is the position.

We use $A$ bits per symbol, when they could be represented in $\log A$ bits, but
this allows us to utilise binary RRR, which we do in the \emph{`Multi-Binary RRR'} Wavelet Tree discussed in Section \ref{sec:multi-bin-rrr}.

\subsection{Multiary Wavelet Tree Variation 2 : Multi-Binary RRR}
\label{sec:multi-bin-rrr}
Like the uncompressed version, bitmaps are created for each symbol and 
concatenated, but the bit-vector is stored in a binary RRR sequence. A query 
then becomes two binary rank queries\footnote{In our implementation we 
pre-calculate the first binary rank queries for each symbol and store it with 
the node.};

	\begin{align}
	rank(c * L - 1, 1) \\
	rank(c * L + i, 1)
	\end{align}

Where $c$ is the symbol we are querying at position $i$,
$c > 0$, and $L$ is 
the original length before concatenation. If $c = 0$ then we say the result of 
the first binary rank query is $0$. The final result of $rank(i, c)$ is 
calculated as the second binary rank query minus the first.

This variation means that we will not need to store a bigger $G$ table to 
accommodate the additional classes when increasing the arity, when compared with
a generalised RRR structure, but does not offer the same sequence compression as
the concatenated bitmaps take more bits than required.

Our third variation stores the symbols (without binary encoding) in a 
generalised RRR structure.

\subsection{Generalised RRR}
\label{sec:gen-rrr}

\DefFig{fig:gen-gtab}{preliminaries/multi-g-table}{1}
	{4-ary RRR Count Table, with example lookup for class $c = 2$, 
	which represents $(2, 2, 0, 1)$, and offset $o = 3$ in a RRR
	sequence.}
	
The purpose of the Generalised RRR structure is to provide $O(1)$ time rank
queries and compression of a sequence of small integers (as opposed to binary 
integers) on the range $[0..A-1]$ for arity $A$. This requires these 
differences:

\begin{itemize}
	\item
		Rather than simply being the number of 1-bits, classes are now 
		considered to be a tuple of $(N^0, N^1, ..., N^{A-1})$ for a given 
		block, where $N^0$ is the number of $0$s, $N^1$ is the number of $1$s, 
		and so on. The RRR sequence still stores classes as a unique integer, 
		though.

	\item
		Rather than having $b \choose N^1$ permutations per class for blocksize 
		$b$, there are now ${b \choose N^0, N^1,...,N^{A-1}} = 
		\prod_{i = 1}^{A} {{b - \sum_{j = 1}^{i - 1} N^j} \choose N^i}
		$ different permutations. Each offset value $o$ therefore requires 
		$\lceil\log {b \choose N^0, N^1,...,N^{A-1}}\rceil$ bits.
		The number of different permutations grows rapidly as arity increases, 
		so our implementation only stored the classes and offsets that we 
		encountered in an attempt to make use of sparsity we can expect in
		the BWT.

	\item
		The $G$ table must also store cumulative ranks for each symbol, per 
		permutation. See Figure \ref{fig:gen-gtab}.

	\item
		Each superblock now has $A$ rank sums of each previous block - a rank 
		sum for each symbol.
\end{itemize}
