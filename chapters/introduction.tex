\section{Introduction}

As our collections of text grow larger, our need to find information in them
and infer patterns and rankings increases, too. Suffix arrays, as first
described by Manber and Myers~\cite{manber1990}, allow a variety of complex
operations to be performed in a time of order $O(p + \log N)$~\cite{manber1990}
where $p$ is the pattern length, and N is the length of the text we are
indexing. There are many areas where suffix arrays are likely the most
appropriate data structure for the task, including:

\begin{itemize}
\item
	Searching for patterns in oriental languages; as some oriental languages
    don't have spaces between certain particles, an inverted file would be
    insufficient
\item
	Pattern discovery and visualisation in using arc diagrams, as proposed by
    Wattenberg~\cite{arc:wattenberg2002}
\item
	Human genome analysis~\cite{genome:abouelhoda2004, genome:flicek2009}
\item
	Other text mining applications such as sentiment classification
\end{itemize}

Due to its performance in these important applications, suffix arrays have been
the focus of intensive research over the past 20 years. On a broad
view\footnote{These will be discussed in more detail in the following sections},
these improvements include several sub-structures such as the Burrows-Wheeler
Transform, which enables \emph{self indexing}; we can discard the original text
and perform fast pattern matching by using a ``backwards search''.

BWT requires a fundamental operation called a ``rank-query''. To improve 
rank-query from $O(N)$ to $O(\log \sigma)$ where $\sigma$ is the alphabet size, we
implement the BWT on a wavelet tree, which are currently binary only. 

The key motivation behind this project arises from the increasing number of
papers, such as ~\cite{yu2009}, which utilise \emph{multi-ary} Wavelet
Trees as a theoretical tool. However no known implementations of multi-ary WTs
exist; this project aims to address this need, and bring theory closer to
practice. It is expected that increasing the supported arity will have benefits
on both the time and space performance of self-indexes which use BWT.

\section{Background}
In 1970 Knuth, Morris and Pratt discovered an algorithm to match patterns in
time proportional to the length of the string~\cite{KMP77, McCreight76}. The
issue with the KMP algorithm is that it must search the whole string to find all
occurrences of it, thus rendering it ineffective for ranking and pattern
discovery. KMP is only useful for
exact matches.

One alternative to KMP for document ranking is the use of inverted files, but
they must work with keywords and are thus inappropriate for many applications,
such as searches on certain oriental languages, and other strings that don't
have a clear definition of keywords (e.g. MIDI). Suffix arrays are also more
efficient than inverted files for searching phrases or partial
patterns~\cite{MN10}.

This was originally
possible with a suffix tree~\cite{McCreight76}, although suffix trees require
three to five times as much space~\cite{manber1990}. A suffix array can require
around $5N$ bytes of memory~\cite{manber1990}, although further improvements can
be made to decrease both time and space requirements.

There have been many additional improvements made to suffix arrays
in recent years. One such improvement is the backwards search algorithm;
since all occurrences of a pattern lie in a contiguous portion of the suffix
array, in earlier implementations we would locate the range that this pattern
lies on by successive binary searches. 

Backwards search utilises the BWT in a series of rank queries, further
improving the query performance considerably ~\cite{CN08, FGM09, FMMN07, GMR06,
MN07:rankselect, MN07:selfindex, MN10, MN06}. One of the most effective data
structs for answering rank queries is the wavelet tree~\cite{CN08, FGM09,
FMMN07, GGV03, MN07:selfindex}. As proposed by Ferragina and Manzini in
~\cite{fmindex:ferragina2000}, when a BWT is stored along with a Wavelet Tree
over the BWT, it is called an \emph{FM-Index}.

Recently, Multi-arity Wavelet Trees have been used in theoretical proofs
~\cite{FMMN07, yu2009}. There are also ways of improving popcount operations and
compressing Wavelet Trees by using a structure called \emph{``RRR''}
~\cite{rrr2007}.
