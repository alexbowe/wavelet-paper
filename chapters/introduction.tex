\section{Introduction}

As collections of text grow larger, our need to find information in them
and infer patterns and rankings increases, too. Suffix arrays, first
described by Manber and Myers \cite{manber1993}, allow a variety of complex 
pattern matching and pattern discovery problems to be performed in optimal time. 
There are many areas where suffix arrays are likely the most appropriate data 
structure for the task, including:

\begin{itemize}
	\item
		Genome analysis~\cite{abouelhoda2004, flicek2009};
	\item
		Searching for patterns in non-word data such as images, multimedia 
		signals and DNA~\cite{culpepper2010}
	\item
		Searching for patterns in oriental languages; as some oriental languages
    	do not have spaces between certain particles, an inverted file would be
    	insufficient;
	\item
		Pattern discovery and visualisation using arc diagrams, as proposed by
    	Wattenberg~\cite{wattenberg2002};
\end{itemize}

Due to its performance in these important applications, the improvement of 
suffix arrays has been the focus of intensive research over the past 20 years. 
\emph{Self-indexing} is one of these improvements. Self indexes support 
fast pattern counting (reporting the number of occurrences of a pattern in a 
text), fast pattern matching (reporting the positions of each pattern 
occurrence) and extraction of arbitrary 
substrings of the original text, including the original text itself. These 
operations in some sense allow us to replace the original text with a self 
index.

\DefFig{fig:overview}{introduction/overview}{1}
	{Overview of the key data structures and their relationship
	with each other. When a Suffix Array and a Burrows-Wheeler 
	Transform (BWT) \cite{burrows1994} are combined, they form an FM-Index 
	\cite{ferragina2000}. FM-Indexes use 
	Backward Searches on the BWT to provide fast pattern 
	matching, counting, and substring extraction operations. Backward
	Search requires rank operations, which are best implemented using
	a Wavelet Tree (WT) \cite{grossi2003}. A WT encodes a string as a hierarchy 
	of bit vectors, 
	which it uses to answer general rank queries using $\log \sigma$ 
	\emph{binary} rank 
	queries. Binary rank queries can be answered in $O(1)$ time when the bit 
	vector is stored as a RRR sequence \cite{raman2007}, which utilise a global 
	table of pre-calculated ranks. The RRR data structure also offers 
	compression of the Wavelet Tree. (we defer a detailed discussion of RRR to
	Section \ref{sec:rrr})}

One such self index is known as the \emph{`FM-Index'}, proposed by
Ferragina and Manzini \cite{ferragina2000}. An FM-Index utilises a 
sparse Suffix Array and the \emph{Burrows-Wheeler Transform} (BWT) \cite{burrows1994} (the two leftmost 
diagrams in Figure \ref{fig:overview}) of the original text, and supports fast 
pattern matching operations by using a \emph{`backward search'}.

Backward searching with a BWT requires a fundamental operation called a 
\emph{`rank query'}, which counts the number of occurrences of a given symbol up 
to a given position in the string.

While a naive implementation of the rank query operation might inspect every 
symbol of the BWT, and do so in $O(N)$ time where $N$ is the length of the BWT, 
it is possible to improve this to $O(\log \sigma)$ where $\sigma$ is the size of 
the alphabet. To do this, we implement a Wavelet Tree \cite{grossi2003} over the BWT (the third 
diagram in Figure 
\ref{fig:overview}). A Wavelet Tree is constructed by encoding the string as a 
hierarchy of bit vectors. These bit vectors are then used to answer rank 
queries by traversing and performing \emph{binary} rank queries\footnote{ Binary rank queries are also known as \emph{popcounts} in other literature.}.

Binary rank queries can be performed in $O(1)$ time when the bit vectors are stored in
a RRR sequence, as proposed by Raman, Raman and Rao \cite{raman2007}. RRR also 
offers compression of the Wavelet Tree.

The key motivation behind this project arises from the increasing number of
papers, such as Ferragina et al. 2007
\cite{ferragina2007}, Yu et al. 2009 \cite{yu2009}, and Barbay et al. 2010
\cite{barbay2010}, which utilise \emph{Multiary} Wavelet Trees as a theoretical 
tool, that is, Wavelet Trees with a branching 
factor greater than two. However no known implementations of Multiary Wavelet 
Trees exist. This thesis aims to address this need, and bring theory closer to 
practice. It was expected that increasing arity would improve the time 
performance of self-indexes which use a BWT.


\subsection{Our Contribution}

The contribution of this paper is the experimental analysis of two types of 
Multiary Wavelet Tree. 
The first of these, to our knowledge, is the first implementation of the Generalised RRR structure, which is used to support rank queries on small 
alphabets, as first suggested by Ferragina et al. \cite{ferragina2007}.

We also propose a new data structure, a Multiary Wavelet Tree that utilises 
Binary RRR using concatenated bitmaps to represent sequences on small alphabets 
as bit vectors.

We discover that although Multiary Wavelet Trees are faster, Generalised RRR 
requires significant memory for the supporting global table in its current form, 
which, depending on the size of the text, may be impractical. We show that our 
alternative data structure, which continues to use Binary RRR, is a practical 
way to make rank queries faster.


\subsection{Roadmap}

The rest of the thesis is organised as follows. Section \ref{sec:prelim} begins 
with a some notation, followed by background on the problem 
domain. We then discuss how each data structure is built and used together.

Section \ref{sec:wt-rrr} provides a description of Binary Wavelet Trees and 
their use of the RRR structure, followed by a discussion in Section 
\ref{sec:multiary} of how our Multiary Wavelet Trees are designed, including 
some implementation details of interest.

In Section \ref{sec:experiments} we describe how we measured the 
time and space performance for each Multiary Wavelet Tree variation,
and provide a rationale of our test dataset. Our results are presented in 
Section \ref{sec:results}, accompanied by a discussion of the apparent trends.

The conclusion follows in Section \ref{sec:conclusion}. We discuss the 
practicality of each Multiary Wavelet Tree variation, and in Section \ref{sec:future} we 
consider how their performance might be improved.

